\section{Ethical Considerations}
Due to the field in which this project is dealing, there are ethical considerations worth noting. Primarily, the issue of data security and autonomy is an important factor when it comes to usage of this network. The datasets that were used were publicly available on Kaggle, but if use of it were to be expanded, it could be repurposed into a data scraping tool for social media platforms such as Twitter. If a program parses through thousands of tweets, there may be problems with sensitive or private information being given to companies, such as the ones that were being referenced in the tweet datasets used. Even without explicitly sharing private information, in the process of data collecting, it is possible to create a “dummy” tied to the account that holds the online information and behaviour of a person for the purposes of targeted advertisement. While the model itself doesn’t have inherent ethical issues, as it just classifies inputs that can usually be classified by a person already, the method by which additional data beyond the public datasets used could lead to questionable ethical practices.