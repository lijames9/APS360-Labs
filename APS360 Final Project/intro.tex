\section{Problem Statement}

In this day and age, where most everyday interactions are digitized, it is absolutely vital to be able to extract essential information from said interactions and use them to your advantage. According to an investigation conducted by McKinsey and Co.,\cite{mckinsey} in 2020 following the COVID-19 crisis, it was noted that the global digitization of customer interactions drastically accelerated from 36\% in December 2019 to 58\% in July 2020 and is still growing to this day. %Similarly, a study conducted in 2020 \cite{screentime} found various interesting statistics, including the increase of Internet usage by up to 50-70\% in households, where over half of that usage is on social media, and that screen time for adults has increased by 5 hours, representing a 60-80\% increase from before the COVID-19 pandemic.

Given the context, the project goal is to create a tweet sentiment analysis classification model using a convolutional neural network (CNN) that would classify the tweet as one of three sentiments: positive, negative or neutral. This model can then be used by other companies by transfer learning, to analyse and classify text data to deduce quality insights regarding their products. Since social media presence is so abundant, this tool will be great for marketing teams looking to gain insights on product quality. 

This is demonstrated by training the model using a large database with a variety of tweets and testing it using a separate database, a variety of tweets mentioning Dell. It is important to discuss why deep learning is necessary for such a task; the ability to learn hierarchical representations and model complex relationships is what it a great candidate for the task. In addition, a CNN is easily able to detect low-level features that include words or word combinations which are then used to learn higher level features such as phrases and expressions with those words. What a CNN offers that is different from an LSTM or RNN is that it does not require time-series data, appropriate for tweets which do not necessarily follow ordinary grammatical and sequential structure. By leveraging these deep learning techniques, the aim is to build a robust model that can accurately classify sentiments in texts, enabling faster and objective-based analysis. \textbf{\textit{Click the following for the illustration: \ref{fig:illustration}}}


%What our project is, why we chose to do this project, why deep learning is appropriate for this type of project, why its interesting (only 2 pts so don't sweat it)

% MAYBE ADD ILLUSTRATION HERE: project idea + architecture as a flowchart maybe?
