\section{Brief Description}

After reviewing comments, discussing feedback from our TA, and critically thinking as a team we decided to pivot the central idea of the project from "Sleep Attribute Classification" to an natural language processing (NLP) based classification problem: "Sentiment Analysis based Classification." The motivation behind our project is to develop a sentiment analysis model that can analyze and classify tweets and reviews. This model, then can be used for transfer learning, as other entities can use the model to analyse and classify their own data to deduce quality information about their products. We demonstrate this by training our model on a large data set from Twitter. After we achieve a satisfactory accuracy (the maximum possible in allocated time) we will test the models predictions on a completely new data set from DELL, that has snippets of conversations and tweets it was tagged in. Following this road map we would be able to showcase exactly how the model can we used in real time.

Our project aims to build an Recurrent Neural Network (RNN) based sentiment analysis model that takes tweets, reviews or any text as input and predicts the sentiment associated. The goal is to classify the sentiment as positive, negative, or neutral, enabling stakeholders to quickly understand the overall sentiment of a the text without having to read the whole text. This can save time and effort of the stakeholders by automating the sentiment extraction process, allowing them to focus on higher-level analysis and decision-making to areas that require attention.

Finally, it is also important to discuss the reasoning of why our team decided to use a "Deep Learning" approach for such a task. The ability to learn hierarchical representations and model complex relationships is what makes deep learning a great candidate for the task of generating sentiment classifications from textual inputs. A RNN can automatically learn complex structures of data capturing both low-level and high-level features simultaneously. This is especially important for a sentiment analysis task as it relies on both the low-level syntactical features as well as the high-level semantics of the text. Additionally, deep learning models prevail at learning from unstructured data, such as text, by learning from large labeled data sets. Finally, from our research we can deduce that deep learning models have demonstrated to perform well in various natural language processing tasks, including sentiment analysis, making deep learning a clear choice for the project.

By leveraging deep learning techniques, we aim to build a robust model that can accurately classify sentiments in texts, enabling faster and objective based analysis.
