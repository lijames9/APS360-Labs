\section{Background and Related Work}

\subsection{Sentiment Analysis using Deep Learning: Financial Market Prediction \cite{mckinsey}} % MAKE CITATION 
\vspace{-1em}
The report from McKinsey highlights that COVID-19 accelerated technological adoption, driving companies across industries towards a permanent digital transformation. It emphasizes that organizations which effectively leverage technology have gained a competitive edge, reshaping business operations and customer interactions. Sentiment analysis using deep learning emerges as a pivotal tool, enabling firms to understand and respond to evolving customer preferences, thus shaping their strategies in this tech-driven landscape.
\vspace{-1em}
\subsection{Sentiment Analysis using Recursive Neural Networks (RecNN) \cite{background1}} 
\vspace{-1em}
The report explores the significant advantages of utilizing Deep Learning for Sentiment Analysis. It delves into the effectiveness of deep neural networks in capturing intricate emotional nuances from textual data. The study emphasizes the model's enhanced ability to extract context-based sentiment, contributing to more accurate sentiment classification in various applications.
\vspace{-1em}
\subsection{Opinion Mining \cite{background2}}
\vspace{-1em}
The report "Sentiment Analysis and Opinion Mining: A Survey" provides an in-depth examination of sentiment analysis techniques, highlighting the growing prominence of deep learning in this field. It explores various methodologies for sentiment classification, emphasizing the advantages of leveraging deep learning models such as neural networks. The survey underscores how deep learning enables more accurate sentiment analysis by capturing complex linguistic nuances and contextual information, leading to improved sentiment classification performance.
\vspace{-1em}
\subsection{Sentiment Analysis: Twitter \& COVID-19 \cite{background3}}
\vspace{-1em}
The report explores the significant advantages of employing deep learning techniques for sentiment analysis. By utilizing advanced neural network architectures, the study demonstrates enhanced accuracy in sentiment classification tasks, particularly in capturing complex emotional nuances. The findings highlight the potential of deep learning to optimize sentiment analysis processes across various domains, underscoring its capacity to uncover deeper insights from textual data.
\vspace{-1em}